% \documentclass[%
\documentclass[11pt,a4paper]{report}
%
% onecolumn,%
% oneside,%
% floats,%
% aps,%
%  prd,%
% nobibnotes,%
% nofootinbib,%
% amsmath,%
% amssymb,%
% amsfonts,%
% amscd,%
%  %   showpacs,%
%   superscriptaddress,%
% % flushbottom,%
% eqsecnum%
% ]{revtex4}
\usepackage{authblk}
\usepackage[T2A]{fontenc} % remove in english version
\usepackage[utf8]{inputenc}
% \usepackage[russian,english]{babel} % remove in english version
\usepackage[english]{babel} % open in english version
\usepackage{graphicx}
\usepackage{xcomment}
\usepackage{bm}
\usepackage{float}
\usepackage{wasysym}
\usepackage{amssymb}
% \usepackage{breqn}

% \usepackage{epsfig}
% \usepackage{amssymb}
% \usepackage{amsmath}
% \usepackage{caption}
% \DeclareCaptionLabelFormat{fig}{\textbf{Figure #2}}
% \DeclareCaptionLabelFormat{ext}{\textbf{Extended Data Figure #2}}
% \captionsetup{labelformat=fig}
% \bibliographystyle{naturemag} %use PR instead
\newcommand{\code}[1]{\texttt{#1}}
\newcommand\cqg{Class. Quant. Grav.}
% \newcommand\apj{Astrophys. J.}
\newcommand\apjs{Astrophys. J. Suppl.}
\newcommand\apjl{Astrophys. J. Letters}
\newcommand\apss{Astrophys. Space Sci.}
\newcommand\jetpl{JETP Letters}
\newcommand\mnras{Mon. Not. of the Royal Astro. Soc.}
\newcommand\aap{Astron. \& Astrophys.}
% \newcommand\nat {Nature}
\newcommand\nature{Nature}
\newcommand\physrep{Phys. Rep.}
% \newcommand\prl{Phys. Rev. Lett.}
% \newcommand\prb{Phys. Rev. B}
% \newcommand\prd{Phys. Rev. D}
\newcommand\science{Science}
\newcommand\sovast{Soviet Astronomy}

% \usepackage{color}
% \definecolor{darkperiwinkle}{rgb}{0.4,0.4,0.5}
% \newcommand{\roland}[1]{\textcolor{darkperiwinkle}{RH: #1 }}

\begin{document}

\title{Errata in \\
R.P.Feynman A.R.Hibbs \\
Quantum Mechanics and Path Integrals}

% \author[1]{Author A}
% \author[1]{Author B}
% \author[1]{Author C}
% \author[2]{Author D}
% \author[2]{Author E}
% \affil[1]{Department of Computer Science, \LaTeX\ University}
% \affil[2]{Department of Mechanical Engineering, \LaTeX\ University}
% {\textit {\{email1,email2,email3,email4,email5\}@xyz.edu}}
%\eid{123@gmail.com}
\author[1,2,3]{S.I.~Blinnikov}
\affil[1]{NRC ``Kurchatov Institute'' -- ITEP, Moscow,  Russia}
% \affil{Alikhanov Institute for Theoretical and Experimental Physics (ITEP) of National Research Center ``Kurchatov Institute'', Moscow, Russia}
\affil[2]{Dukhov Research Institute of Automatics (VNIIA), Moscow,  Russia}
\affil[3]{Kavli IPMU (WPI), Tokyo University, Kashiwa, Japan}
\affil[ ]{email: {\tt Sergei.Blinnikov@itep.ru}}

\author[4]{N.V.~Nikitin}
\affil[4]{National Research Nuclear University MEPhI (Moscow Engineering Physics Institute)}



\maketitle

\begin{abstract}
A list of Errata is given for the Russian edition R.P.Feynman A.R.Hibbs Quantum Mechanics and
Path Integrals, Mir, Moscow, 1968. Hereafter denoted ФХ.
English original (1965) is denoted FH. Many typos in FH are corrected as well.

\end{abstract}


% \section{Errata}
% \label{sec:errata}

\chapter{The Fundamental Concepts of Quantum Mechanics}

\chapter{The Quantum-mechanicaL Law of Motion}

\section{The Classical Action}

\subsection*{equation 2.9}

Equation (2.9) both in FH and ФХ

$$
S_{\rm cl} = \frac{m\omega}{2\sin\omega T}  (x_a^2 + x_b^2) \cos \omega T - 2x_a x_b \eqno (2.9)
$$
should read
$$
S_{\rm cl} = \frac{m\omega}{2\sin\omega T} \left[ (x_a^2 + x_b^2) \cos \omega T - 2x_a x_b \right] \eqno (2.9)
$$


\section{The Quantum-mechanical Amplitude}

\section{The Classical Limit}

\section{The Sum over Paths}

\section{Events Occurring in Succession}

\section{Some Remarks}


\chapter{ Developing the Concepts with Special Examples }

\section{The Free Particle}

FH p. 43. Eq. (3.4)
\[
 \left( \frac{2\pi i\hbar\epsilon}{m}  \right)^{-2/2}
\]
and in Russian ФХ
\[
 \left( \frac{2\pi i\hbar\epsilon}{m}  \right)^{-1/2}
\]
both should read
\[
 \left( \frac{2\pi i\hbar\epsilon}{m}   \right)^{-1}
\]
and in arguments to $\exp$ in (3.4), (3.5), and two formulas after (3.5),
all factors like
\[
 \frac{m}{2i\hbar \cdot n\epsilon}
\]
should read like
\[
 \frac{im}{2\hbar \cdot n\epsilon}
\]
for $n=1,\, 2,\, 3,\, \ldots n$, then we arrive to formula (3.3).

\section{Diffraction through a Slit}

FH p. 47.

At $t = 0$ the particle starts from the origin $x = 0$.

A few lines lower:

finding the particle at an additional displacement $x$ from the position $x_0$, where $x_0$ is nonzero.

This leads to some confusion in equation (3.19): one should remember that $x$ has changed its zero point.
It would be better to use another notation, say, $x'$ in (3.19) with the zero point for $x'$ at $x_0$.

\subsection*{equation 3.27}

ФХ p. 64 eq.~(3.27)

\begin{eqnarray*}
 \psi(x) =   \sqrt{\frac{m}{2\pi i \hbar}} \left(T + \tau + T \tau \frac{h i}{m b^2}\right)^{-1/2} \times \\
 \times \exp\left[\frac{im}{2\hbar}\left(v_0^2T + \frac{x^2}{\tau} \right)
 + \frac{(m^2/2\hbar^2\tau^2)(x-v_0/\tau)}{(m/\hbar)(i/T+i/\tau) - 1/b^2) } \right] .
 \qquad (3.27)
\end{eqnarray*}

must read, replacing $h$ by $\hbar$ in the 1st line, and then $-v_0/\tau$ by  $-v_0\tau$ and omitting the bracket after $1/b^2$,

\begin{eqnarray*}
 \psi(x) =   \sqrt{\frac{m}{2\pi i \hbar}} \left(T + \tau + T \tau \frac{\hbar i}{m b^2}\right)^{-1/2} \times \\
 \times \exp\left[\frac{im}{2\hbar}\left(v_0^2T + \frac{x^2}{\tau} \right)
 + \frac{(m^2/2\hbar^2\tau^2)(x-v_0\tau)}{(m/\hbar)(i/T+i/\tau) - 1/b^2 } \right] .
 \qquad (3.27)
\end{eqnarray*}

It is OK in English original, which we denote FH, note that $\hslash$ is consistently used
instead of $\hbar$ there.

\subsection*{equation 3.28}


Eq.~(3.28)
$$
P(x) dx = \frac{m}{2\pi \hbar T} \frac{b}{\Delta x}
           \exp\left[\frac{-(x-v_0\tau)^2}{(\Delta x)^2}  \right] . \eqno (3.28)
$$
must read
$$
P(x) dx = \frac{m}{2\pi \hbar T} \frac{b}{\Delta x}
           \exp\left[\frac{-(x-v_0\tau)^2}{2(\Delta x)^2}  \right] . \eqno (3.28)
$$
The same equation in the English original FH p.51 misses ``2'' before $(\Delta x)^2$.
% \textit{Must be checked! SB done}

\subsection*{equation 3.33}


Eq.~(3.33)
$$
 P(\mbox{any} \; x) = \frac{m}{2\pi\hbar T} b \sqrt{\pi} . \eqno (3.33)
$$
must read in ФХ and FH:
$$
 P(\mbox{any} \; x) = \frac{m}{2\pi\hbar T} b \sqrt{2\pi} . \eqno (3.33)
$$

\section{Results for a Sharp-edged Slit}

\section{The Wave Function}

\section{Gaussian Integrals}


ФХ p.~74

Такое положение представляется характерным для различных методов вычисления интегралов по траекториям;
при помощи общих приемов могут быть получены многие результаты, однако оказывается, что часто не удается полностью определить экспоненциальный коэффициент.

экспоненциальный $\to$ предэкспоненциальный

 a multiplying factor in FH is OK.


\end{document}
